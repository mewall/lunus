\documentstyle[12pt]{article}
\begin{document}
The code for obtaining and analyzing diffuse scattering data is built
according to the specifications of the makefile '{\tt ./src/Makefile}'
under the home directory for the package.  The code was developed on a
SGI workstation running Irix 5.2, so the makefile may need to be
modified to build the software on different systems.  The shell
environment variable {\tt C\_HOME}, which specifies the home directory
for the package, needs to be defined for the makefile to work.  For
example, the command ``{\tt setenv C\_HOME $\sim$/c}'' would need to be
executed before 
running the makefile if the home directory were {\tt
$\sim$/c}.

Main source code is located in the {\tt ./src} directory -- all files have
the suffix '{\tt .c}'.  For the most part, this code only parses the shell
command-line arguments, and makes library calls.  Allocation and
initialization of data structures, file i/o, and number crunching are,
for the most part,  handled by library calls.  A list of the head of
all of the main source code is in {\tt c-src-head.txt}.

Library source and object code is in the {\tt ./lib} directory.  All
library file names begin with the letter '{\tt l}' to distinguish them
from the main routines.  Source code file names end in '{\tt .c}',
while object code file names end in '{\tt .o}'.  The library '{\tt
libmw.a}' contains all of the object files.  A list of the head of all
of the library source code is in {\tt c-lib-head.txt}.

Executable binaries are in the {\tt ./bin} directory.  Programs are executed
by typing the file name at the shell prompt.  If arguments are
improperly specified, a program will terminate with a short
description of their use.  A quick way to find out how to use a
program is to simply type its name at the prompt -- longer
descriptions are found at the top of the main source code by the
same name as the executable.

The master header file is in the {\tt ./include} directory, and is
called '{\tt mwmask.h}'.  In this header, standard header files are
included, initialization defaults are defined, and data structures are
defined.  Presently, it is neccessary to specify some experimental
parameters in this header file, so that each time data from a new
experiment is analyzed, the parameters must be changed in the mwmask.h
file, and the code must be recompiled.  The relevant parameters are
those enclosed in a series of {\tt \#ifdef/\#else} statements near the
top of the file, and are selected by defining the relevant label for
the codename of the data set at the top of the file.  To make this
more friendly, the code which mainly needs to be modified is that
dealing with mapping diffraction images to a lattice, such as {\tt
map2lat} and {\tt genlat}, although other routines may need to be
changed, too.
\end{document}
